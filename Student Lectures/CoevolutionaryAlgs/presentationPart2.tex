\documentclass{beamer}
%
% Choose how your presentation looks.
%
% For more themes, color themes and font themes, see:
% http://deic.uab.es/~iblanes/beamer_gallery/index_by_theme.html
%
\mode<presentation>
{
  \usetheme{default}      % or try Darmstadt, Madrid, Warsaw, ...
  \usecolortheme{default} % or try albatross, beaver, crane, ...
  \usefonttheme{default}  % or try serif, structurebold, ...
  \setbeamertemplate{navigation symbols}{}
  \setbeamertemplate{caption}[numbered]
} 

\usepackage[english]{babel}
\usepackage[utf8x]{inputenc}
\usepackage{multimedia}
\usepackage{media9}
%\usepackage[dvipdfmx]{media9} % only for latex->dvipdfmx
\usepackage{hyperref}

\title[CoEvAlgos!]{Co-evolutionary algorithms}
\author{Abhishek \& Suhas}
\institute{Purdue University}
\date{2016 October 3rd}

\begin{document}

\begin{frame}
  \titlepage
\end{frame}

% Uncomment these lines for an automatically generated outline.
\begin{frame}{Outline}
  \tableofcontents
\end{frame}

\section {IPD walkthrough}
\begin{frame}{Iterated Prisoner's Dilemma - A code walkthrough}

\end{frame}
\note[itemize]{
\item An alternative way of putting it is using the Darwinian ESS simulation. In such a simulation, tit for tat will almost always come to dominate, though nasty strategies will drift in and out of the population because a tit for tat population is penetrable by non-retaliating nice strategies, which in turn are easy prey for the nasty strategies. Richard Dawkins showed that here, no static mix of strategies form a stable equilibrium and the system will always oscillate between bounds.
\item http://www.prisoners-dilemma.com/results/cec04/ipd_cec04_full_run.html
}

\section{Possible evolved strategies}
\begin{frame}{Possible evolved strategies}
    \begin{itemize}
        \item Without agents referring to each other's history, Nash equilibria are the only possible solutions.
            \includegraphics[width=300pt]{low_nash}
            \caption{Nash equilibria only stable strategies when agents have no memory\\
            Source: F. Seredynski, Loosely Coupled Genetic Algorithms}
        \item With memory, much more robust strategies like tit-for-tat or win-stay-lose-switch can emerge.
            \includegraphics[width=300pt]{Lookerup_performance}
            \caption{Starting moves + 2 recent move based GA-evolved-optimal look up table\\
            \note{Source: http://mojones.net/evolving-strategies-for-an-iterated-prisoners-dilemma-tournament.html}}
    \end{itemize}
\end{frame}

\section{Theory --> real world}
\begin{frame}{Prisoner's dilemma in real life, interesting effects}
    \begin{itemize}
        \item \textbf{Tragedy of the commons}
        \item \textbf{Advertising} investments and their payoffs among competitors.
        \item \textbf{Doping} of competing athletes in sport.
        \item \textbf{Disarmament} of weapons; Cold War.
    \end{itemize}
\end{frame}

\begin{frame}{Interesting effects}
    \begin{itemize}
        \item 
    \end{itemize}
\end{frame}

\begin{frame}{Rock-Paper-Scissors stalemate}
    \movie[width=8cm,height=6cm,poster,showcontrols]{Three minutes of chaos and elevator music}{Visualizing.mp4}\\
    Source \href{https://www.youtube.com/watch?v=FTBUIJwlVBg}{Visualizing coevolution in dynamic fitness landscapes }

\end{frame}

\note[itemize]{
\item In recent years, evolutionary models have been used to assist decision making in applied settings and find solutions to problems such as optimal product design and service portfolio diversification.
Source:  Baltas, G., Tsafarakis, S., Saridakis, C.  Matsatsinis, N. (2013). Biologically Inspired Approaches to Strategic Service Design: Optimal Service Diversification Through Evolutionary and Swarm Intelligence Models. Journal of Service Research 16 (2), 186-201

\item unfair ZD strategies are not evolutionarily stable. The key intuition is that an evolutionarily stable strategy must not only be able to invade another population (which extortionary ZD strategies can do) but must also perform well against other players of the same type (which extortionary ZD players do poorly, because they reduce each other's surplus)
Source:Adami, Christoph; Arend Hintze (2013). "Evolutionary instability of Zero Determinant strategies demonstrates that winning isn't everything": 3. arXiv:1208.2666free to read.

\item Le and Boyd[17] found that in such situations, cooperation is much harder to evolve than in the discrete iterated prisoner's dilemma. The basic intuition for this result is straightforward: in a continuous prisoner's dilemma, if a population starts off in a non-cooperative equilibrium, players who are only marginally more cooperative than non-cooperators get little benefit from assorting with one another. By contrast, in a discrete prisoner's dilemma, tit for tat cooperators get a big payoff boost from assorting with one another in a non-cooperative equilibrium, relative to non-cooperators. Since nature arguably offers more opportunities for variable cooperation rather than a strict dichotomy of cooperation or defection, the continuous prisoner's dilemma may help explain why real-life examples of tit for tat-like cooperation are extremely rare in nature (ex. Hammerstein[18]) even though tit for tat seems robust in theoretical models.
Le S, Boyd R (2007). "Evolutionary Dynamics of the Continuous Iterated Prisoner's Dilemma". Journal of Theoretical Biology. 245 (2): 258–267. doi:10.1016/j.jtbi.2006.09.016. PMID 17125798. 
}

\end{document}
